\documentclass[twocolumn]{article}

%\usepackage[iso]{umlaute}
%\usepackage{german}
\usepackage{graphicx}
\setlength{\parindent}{0cm}
\setlength{\parskip}{1ex}
\setlength{\columnsep}{25pt}

\textwidth=17cm
\textheight=23cm
\setlength{\unitlength}{0.5cm}
\setlength{\parindent}{0.0cm}
\setlength{\parskip}{1ex}
\raggedbottom
\sloppy
%\addtolength{\evensidemargin}{-5cm}
\addtolength{\oddsidemargin}{-1.5cm}
\addtolength{\topmargin}{-2cm}

\sloppy

% Your name
\author{Trung Nguyen\\ Technische Universit\"at M\"unchen}

\title{Seminar Cloud Computing \\
       {\bf From Concept to Production: Deploying TinyML in Industry}
}

% Date of your talk
\date{November 2024}

\usepackage{hyperref}

\begin{document}

\maketitle

\begin{abstract}
In recent years, Artificial Intelligence (AI) and Machine Learning (ML) have received tremendous amount of attention in both industry and research world. However, conventional Machine Learning demands high computing capability which limits its usage to only larger computing units. The pardigm shift to Tiny Machine Learning (TinyML) is revolutionizing industries by enabling the deployment of machine learning models on low-power, resource-constrained devices. Being one of the most rapid developing field of Machine Learning, TinyML promises to benifits multiple industries. However, building a production-ready tinyML system poses different unique challenges. In this paper, we explore the key obstacles faced when developing and deploying TinyML models in production environments, including model optimization, hardware limitations, software integration, and maintaining performance in real-world conditions. Additionally, we present real-world use cases of TinyML in industrial settings, showcasing its transformative impact. We also discuss practical approaches and strategies presented by recent researches \cite{ren_tinyol_2021} to overcome these challenges, providing insights into how TinyML systems can be successfully scaled and implemented in production.
\end{abstract}

% \section defines numbered parts of the paper with titles
% there also are \subsection and \subsubsection
\section{Introduction}
\label{introduction}


Traditional Machine Learning Models, especially Deep Learning Models typically require substantial amount of computing capability to operate effectively. These models are often trained on powerful Graphics Processing Units (GPUs) and produce large models ranging from tens or hundreds of gigabytes (GB) down to smaller models in the range of 10 to 100 megabytes (MB). However, the memory requirements during runtime for these models far exceed what microcontrollers (MCUs) can handle.
The pardigm shift to TinyML is driven by the prevailing number of Microcontroller Units (MCU) currently circulating in the industry. According to a recent report \cite{noauthor_microcontroller_nodate,grandviewresearch_research_2023}, as of 2021, around 31 billion MCUs were shipped worldwide annually. The MCU market size is projected to increase in upcoming years \cite{noauthor_microcontroller_nodate}. This creates a big incentive for researchers and industry players to put effort into developing the technology further.
TinyML aims to enable the operation of ML model to run on energy-, and memory-constraint devices by limiting communication overhead with better suited architecture design and applying different compressing techniques such as: quantization and pruning. Although various organizations have been putting efforts into ML project, merely 13\% of them took off to production. 

In this paper, we aim to highlight what is required to transition a TinyML project into production and generate value for the industry. In Chapter~\ref{tinyml_overview}, we provide an overview of the key concepts, techniques, and the development pipeline of TinyML. In Chapter~\ref{prod_tinyml}, drawing on recent research, we explore the challenges of this development process and present approaches to address them. Next, in Chapter~\ref{use_cases}, we discuss real-world use cases of TinyML in industrial environments. Finally, in Chapter~\ref{future_of_tinyml}, we examine the future prospects of TinyML.

\begin{figure}
	\centerline{
	\includegraphics[width=1\columnwidth]{resource/tinyml_deployment.pdf}
	}
	\caption{The caption explaining what can be seen in the image/figure.
	Readers often read captions first if they do not have much time. Thus,
	it is important to find a good short explanation.}
	% A label to allow refering to this figure in the text.
	\label{TUM}
\end{figure}

\section{TinyML Overview} 
\label{tinyml_overview}

\subsection{Key Concepts and Techniques}



\subsection{TinyML pipeline}


\begin{equation}
a^2 + b^2 = c^2
\label{Pythagoras}
\end{equation}

Again, refering to this equation is easy (see Eq.~\ref{Pythagoras}).
If you do not need numbering for equations, use the {\em displaymath}
environment:

\begin{displaymath}
x_{1,2} = \frac{-b \pm \sqrt{b^2-4ac}}{2a}\\
\end{displaymath}


\section{Enablements of TinyML in Industrial Setting}
\label{prod_tinyml}

\begin{quote}
	``I think there is a world market for maybe five computers.''
	(T.J. Watson, IBM, 1943)
\end{quote}

The rest of the work (especially all the regular text) must be
written/phrased by you. If you write about some results or fact
stated in another paper, you should refer to it.
The `Analytical Engine'' --- a mechanical calculation machine ---
created by Charles Babbage in the year 1838 was based on the decimal
system
% use \cite to refer to papers from seminarpaper.bib
% this file is processed by bibtex, and it automatically adds numbering
\cite{hussein_original_2024, paul_rethinking_2021, de_prado_robustifying_2020,ren_synergy_2021,roshan_adaptive_2021}.


\section{Use Cases of TinyML in Industrial Setting}
\label{use_cases}

Enumerations using bullet points:

\begin{itemize}
	\item 	Agriculture
	\item 	Environmental Monitoring
	\item 	Industrial predictive maintenance
	\item 	Edge AI and Autonomous Systems
\end{itemize}


\section{Future of TinyML }
\label{future_of_tinyml}



\section{Conclusion}
\label{conclusion}



% Put citations from bibtex into References section which were not
% explicity cited.
\nocite{hussein_original_2024,paul_rethinking_2021}


\bibliographystyle{plain}
% Literature sources are to be found in seminarpaper.bib
\bibliography{seminarpaper}
\end{document}
